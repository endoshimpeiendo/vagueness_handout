%%%%%%%%%%%%%%%%%%%%%%%%%%%%%%%%%%%%%%%%%%%%%%%%%%%%%%%%%%%%%%%%%%%%%%
% How to use writeLaTeX:
%
% You edit the source code here on the left, and the preview on the
% right shows you the result within a few seconds.
%
% Bookmark this page and share the URL with your co-authors. They can
% edit at the same time!
%
% You can upload figures, bibliographies, custom classes and
% styles using the files menu.
%
% If you're new to LaTeX, the wikibook is a great place to start:
% http://en.wikibooks.org/wiki/LaTeX
%
%%%%%%%%%%%%%%%%%%%%%%%%%%%%%%%%%%%%%%%%%%%%%%%%%%%%%%%%%%%%%%%%%%%%%%
\documentclass{tufte-handout}

%\geometry{showframe}% for debugging purposes -- displays the margins

\usepackage{amsmath}
\usepackage{amssymb}
\usepackage[affil-it]{authblk}
\usepackage{stmaryrd}
\usepackage{marvosym}
\usepackage[normalem]{ulem}
\usepackage{soul}
\usepackage{tabularx}
\usepackage{ marvosym }
\usepackage{enumitem}

\newtheorem{definition}{Definition}

% Set up the images/graphics package
\usepackage{graphicx}
\setkeys{Gin}{width=\linewidth,totalheight=\textheight,keepaspectratio}
\graphicspath{{graphics/}}

\title{{\bf{A Dimensional View towards Vagueness} } % Title
\thanks{A talk for Logic and Engineering of Natural Language Semantics 15 (LENLS 15), Hiyoshi Campus of Keio University, Yokohama, Japan. }
}
 % Subtitle

\author{\textsc{Shimpei Endo
\thanks{Master of Logic, ILLC, Unviersity of Amsterdam.
My main interests are metaphysics and logic(s), particularly, modal realism and semantics for non-classical modal logics.
\Letter \url{endoshimpeiendo@gmail.com}
}
}
}

\date{November 14, 2018 (Wed) 13:00-13:30 (incl. discussion)} % Date

%----------------------------------------------------------------------------------------

\begin{document}

\maketitle % Print the title section

\begin{abstract}
We should see \emph{vagueness} from a \emph{dimensional} perspective.
I argue that my solution is good but do not claim that others are wrong.
Rather, my dimensional view provides a formal platform on which discussions make sense.
\end{abstract}

\newthought{Vagueness and sorites paradox revisited.}
Most verbal expressions are vague.
You need to face vagueness at least when you want to analyze languages via logic. An ordinal predicate such as ``is bald'' or ``is a heap'' leads to a well known paradox called \emph{sorites paradox}.

\newthought{What is Sorites paradox
\footnote{Also known as paradox of heaps, paradox of baldman or \emph{little-by-little argument}. Although each of these clauses seems to be plausible, combining them leads to a contradiction.
}
?}

\begin{description}
\itemsep0em
\item	[Obviously bald case:] $M \vDash B(0)$. A man with no hair is surely bald
\footnote{
$M$ is an arbitrary model. $M \vDash \phi$ is read as $\phi$ holds under a model $M$.
We employ a language which only contains a baldness property $B$. $B(x)$ is read as "$x$ is bald".
}.
\item	[Obviously non-bald case:] $M \vDash \neg B (2,000,000)$.
\emph{A man with 200000 hairs is surely \emph{not} bald.}

\item	[\emph{Tolerance Principle}:] If $M \vDash x \sim_{B} y$ and $M \vDash B(x)$, then $M \vDash B(y)$.
\emph{If two things $x$ and $y$ are indifferent (i.e. similar) to each other with respect to a property $B$, then $B(x)$ promises $B(y)$.
\footnote{
With $B$ a property,
$x \sim_{B} y$ iff $\vDash B(x) \Leftrightarrow \vDash B(y)$ read as "x is similar to y with respect to B-ness".
Its negation is defined as follows: $x \not \sim_{B} y$ if and only if  $\vDash B(x) \not \Leftrightarrow \vDash B(y)$.
}
}

\item	[Bridge:] $M \vDash \exists 0 \exists 1, ... , \exists i_{\in I}, ... \exists 2,000,000
:
0$
$\sim_{B} 1 \sim_{B} ... \sim_{B} 1,999,999 \sim_{B} 2,000,000$.
\emph{There exists a stream of similarity relation from a small enough ($0$) to a large enough ($2,000,000$).}

\noindent\rule{10cm}{0.4pt}

\item [Unwelcome conclusion] $\vDash \neg B (2,000,000)$  .
\emph{A man with 20000 hairs turns out to be bald.}
$\blacksquare$

\end{description}

\newthought{Many solutions have been suggested} for sorites paradox.
To name a few popular ones:
\emph{Epistemicists} (e.g. Williamson) points out that tolerance assumption is questionable: i.e. there exists a threshold but we have epistemic limitation to know where it is.
\emph{Supervaluationists} (e.g. Fine) rather revises the \emph{logic} working behind it, accepting truth-value \emph{gap}.

\newthought{See things dimensionally.}
We are already familiar with the dimensional perspective.
We see things not directly in a three-dimensioal structure but in a two-dimensinoal structure (as our retina does).

\newthough{Formalization.}

\newthought{Let us do some baseball.}

\newthought{Return to Sorites.}

\pagebreak

\bibliographystyle{plain}
\bibliography{Mendeley}
\end{document}
