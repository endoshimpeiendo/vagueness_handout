%%%%%%%%%%%%%%%%%%%%%%%%%%%%%%%%%%%%%%%%%%%%%%%%%%%%%%%%%%%%%%%%%%%%%%
% How to use writeLaTeX:
%
% You edit the source code here on the left, and the preview on the
% right shows you the result within a few seconds.
%
% Bookmark this page and share the URL with your co-authors. They can
% edit at the same time!
%
% You can upload figures, bibliographies, custom classes and
% styles using the files menu.
%
% If you're new to LaTeX, the wikibook is a great place to start:
% http://en.wikibooks.org/wiki/LaTeX
%
%%%%%%%%%%%%%%%%%%%%%%%%%%%%%%%%%%%%%%%%%%%%%%%%%%%%%%%%%%%%%%%%%%%%%%
\documentclass{tufte-handout}

%\geometry{showframe}% for debugging purposes -- displays the margins

\usepackage{amsmath}
\usepackage{amssymb}
\usepackage[affil-it]{authblk}
\usepackage{stmaryrd}
\usepackage{marvosym}
\usepackage[normalem]{ulem}
\usepackage{soul}
\usepackage{tabularx}
\usepackage{ marvosym }
\usepackage{enumitem}
\usepackage{setspace}
\usepackage{bm}

\newtheorem{definition}{Definition}

% Set up the images/graphics package
\usepackage{graphicx}
\setkeys{Gin}{width=\linewidth,totalheight=\textheight,keepaspectratio}
\graphicspath{{graphics/}}

\title{{\bf{A Dimensional View towards Vagueness} } % Title
\thanks{A talk for Logic and Engineering of Natural Language Semantics 15 (LENLS 15), Hiyoshi Campus of Keio University, Yokohama, Japan. }
%\thanks{The latest version of this handout \url{www.something.org} and the draft \url{www.otherthings.org} are available online and welcoming comments and suggestions!}
}
 % Subtitle

\author{\textsc{Shimpei Endo
\thanks{Master of Logic, ILLC, Unviersity of Amsterdam.
My main interests are metaphysics and logic(s), particularly, modal realism and semantics for non-classical modal logics.
\Letter \url{endoshimpeiendo@gmail.com}
}
}
}

\date{November 14, 2018 (Wed) 13:00-13:30 (incl. discussion)} % Date

%----------------------------------------------------------------------------------------

\begin{document}

\maketitle % Print the title section

\begin{abstract}
We should see \emph{vagueness} from a \emph{dimensional} perspective.
I argue that my solution is good but do \emph{not} claim that others are wrong.
Rather, my dimensional view provides a formal platform on which the disputes are possible.
\end{abstract}

\newthought{Vagueness revisited.}
Most verbal expressions are vague.
 An ordinal predicate such as ``is bald'' or ``is a heap'' leads to a paradox known as \emph{sorites paradox}.
\footnote{Also known as paradox of heaps, paradox of baldman or \emph{little-by-little argument}. Combining these seemingly plausible assumtoions leads to a contradiction. See:
\cite{Hyde2011-HYDTSP}
%\cite{Sainsbury1995}
}
\footnote{Sorites paradox matters when you analyze natural languages via logic or logical structure in natural languages. }


\begin{description}
\itemsep0em

\item	[Obviously non-bald case:] A man with 2,000,000 hairs is surely \emph{not} bald.

\item	[Obviously bald case:] A man with no hair is surely bald

\item	[Tolerance Principle:]
Pulling a single hair does not make anyone any non-bald person into a bald one.
\begin{spacing}{0}

\noindent\rule{10cm}{0.4pt}
\end{spacing}

\item [Unwelcome conclusion:]
A man with 2,000,000 hairs is bald.
$\blacksquare$

\end{description}

\newthought{My solution: See things dimensionally.}
\footnote{Many solutions have been suggested.
I do not argue that they are all wrong and mine is the only possible or the most plausible one. Instead, my dimensional understanding is expected to describe and embrace these different opinions.}
We are already familiar with the dimensional perspective.
We see things not directly in a three-dimensioal structure but in a two-dimensinoal structure (as our retina does).


\newthought{Formalization.}
Technically speaking, absence and abundance of information are written in terms of \emph{projection functions} on a dimensional structure.

\begin{definition}[Dimensional structure]
Let $X_i$ a space (set).
A dimensional structure $M$ is defined as
$M = \prod_{i \in I} X_i.$
\end{definition}

\begin{definition}[Predicates and objects]
Let $P \in PRED$ be a predicate and $o \in OBJ$ be an object.
Within a dimensional structure $M$,
a predicate $P$ is a subset of $M$, written as $\llbracket P \rrbracket^{M} \subseteq M$. An object $x$ is also a subset of $M$, $o \subseteq M$.
\end{definition}

\begin{definition}[Projection]
\footnote{There are many other possible (and expressive) projections (cf. projective geometry). This simple one is used just for  the sake of explanation. }
Let $x = \{x_0 \in X_0, x_1 \in X_1, ... , x_j \in X_j, ...\}$.
Consider a dimensional structure $M^{i} = \prod_i X_i$.
Pick an arbitrary $j \in I$.
A projection $f_j: M \mapsto \Downarrow_j M$ returns $f_{j}(x)= \Downarrow_{j}x = \{x_0 \in X_0, x_1 \in X_1, ... , x_{j-1} \in X_{j-1}, x_{j+1} \in X_{j+1}, ...\}$.
\end{definition}

\begin{definition}[Evaluation]
$M \models P(x)$ if and only if $x \subseteq \llbracket P \rrbracket$.
\end{definition}

\newthought{Demonstration: baseball.}
\footnote{
\begin{tabular}{lllllll}
$\mathbf{M}$        & P1 & P2 & P3 & P4 & P5 & \textbf{D1}  \\
Fileding & +  & +  & +  & ?  & -  &     \\
Hitting  & +  & -  & ?  & +  & -  &     \\
Running  & +  & +  & +  & ?  & -  &     \\
\textbf{D2}       &    &    &    &    &    &
\end{tabular}

\vspace{2mm}
\hspace{4mm}
{\Large $\Downarrow_{D^*}$}
\vspace{2mm}

\begin{tabular}{llllll}
$\Downarrow \mathbf{M}$ & P1 & P2 & P3 & P4 & P5  \\
In  total                  & +  & ?  & +  & +  & -
\end{tabular}
}

We reach vagueness when we consider too little (\emph{absence})
Suppose a great batter but horrible field player. If we evaluate her/him as a whole, according to our function, s/he is both good and non-good.
 and when we consider too much (\emph{abundance}).
Our (once fixed) evaluation is often cancelled after considering another perspective. Imagine a DH player whom you have never seen on the field. We can conclude that s/he is good without knowing his fielding ability (our function allows such) but once you start caring about tools (i.e. perspectives) ignored, your evaluation becomes more vague.







%\newthought{Return to Sorites.}
%We may consider too little: the number of hair is just one of many possible \emph{dimensions}, which can be insufficient to determine bald or non-bald by itself. In the setting of paradox, the number of hair is just used to specify each object. We need to consult extra dimensions to evaluate each object's baldness.
%Or, we may consider too much: adding new dimensions does not necessarily help to expell vagueness.
%Obvious cases suppose that for these apparent cases (no hair at all or apparetnly many hairs) for a given structure $M^i$ containing $i$ dimensions.
%However, such a set $M^{i}$ does not necessarily offer a sufficient and non-confusing information for less ideal cases between $0$ and $2000000$.
%Note that the dimensional view is compatible with the bridge clause.
%We can keep that for each $n$, $M^{i} \models n \sim_B n+1$ (bridge).
%However, it is a different story to have a complete and sound structure of dimensions.

\newthought{Return to Sorites.}
Construct $\mathbf{M = D^* \times D^\#}$, with $\mathbf{D^*}$ evaluating by a certain person  and $\mathbf{D^\#}$ specifying the number of hair.

\begin{table}
\centering
\begin{tabular}{llllllllllllllll}
$\mathbf{M=}$ \\$ \mathbf{D^* \times D^\sharp}$    & 0 & 1 & 2 & ... & l & l+1 & ... & m & m+1 & ... & n & n+1 & ... & \begin{tabular}[c]{@{}l@{}}2,000,000\end{tabular} & $\mathbf{D^\#}$  \\
You  & + & + & + & ... & + & +   & ... & + & +   & ... & + & +   & ... & -                                                   &                            \\
Me   & + & + & + & ... & + & +   & ... & + & +   & ... & + & ?   & ... & -                                                   &                            \\
Her  & + & + & + & ... & + & ?   & ... & ? & -   & ... & - & -   & ... & -                                                   &                            \\
$\mathbf{D^*}$ &   &   &   &     &   &     &     &   &     &     &   &     &     &                                                     &
\end{tabular}

{\Large $\Downarrow_{D^*}$}



\begin{tabular}{lllllllllllllllllll}
$\mathbf{\Downarrow_{D^*}M}$       & 0 & 1 & 2 & ... & l & l+1 & ... & m & m+1     & ... & n       & n+1     & ... & 2,000,000 &  & $\mathbf{D^{\#}}$ &  &   \\
Us  & + & + & + & ... & + & +   & ... & + & ? & ... & ? & ? & ... & -         &  &   &  &
\end{tabular}

\end{table}

\begin{description}
\itemsep0em

\item	[Obviously non-bald case:] $M \models \neg B (x_{2,000,000})$
\item	[Obviously bald case:] $M \models B (x_{0})$

\item	[Tolerance Principle:] There are \emph{no} $x_{n}$ and $x_{n+1}$ such that $M \models B(x_{n}) $ and $M \models \neg B (x_{n+1})$

\begin{spacing}{0}

\noindent\rule{10cm}{0.4pt}
\end{spacing}

\item [Unwelcome conclusion avoided:]
A person with 2,000,000 hairs does not have to be bald.
$\square$
\end{description}


Obvious bald/non-bald objects $x_0$ and $x_{2,000,000}$ are obviously so
for $x_0 \subseteq \llbracket + \rrbracket^{M}$ and $x_2,000,000 \subseteq \llbracket - \rrbracket^{M}$.
The same holds even in the more ``limited'' model $\Downarrow M$.
\footnote{$+$ is a shorthand for $B$ and $-$ for $\neg B$. }
\emph{Tolerance still holds.} Tolerance prohibits suddenly changing from bald to non-bald.
We can construct bald $x_0$ and non-bald $x_{2,000,000}$
by putting $\llbracket ? \rrbracket$ in between.

\newthought{Previous attempts seen dimensionally!}
Previous solutions are special variants of our dimensional view.
For example, epistemicists \cite{Williamson1994} highlights our epistemic ignorance, corresponding to ``too little to care'' or ``too much to care''.
For another instance, supervaluationists
\cite{Keefe2000}
 would consider our $?$ as their ``truth value gap''.


\pagebreak

\bibliographystyle{plain}
\bibliography{Mendeley}
\end{document}
